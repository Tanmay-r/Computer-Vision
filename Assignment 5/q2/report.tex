\title{Assignment 3: Question 2: CS 763, Computer Vision}
\author{}

\documentclass[11pt]{article}

\usepackage{amsmath,cancel}
\usepackage{amssymb}
\usepackage{hyperref}
\usepackage{ulem,color}
\usepackage[margin=0.5in]{geometry}
\begin{document}
\maketitle


A very beautiful aspect of compressive sensing is the rigorous mathematical basis - in the form of concrete error bounds on reconstruction results. While using regularization to solve ill-posed problems is a known technique in computer vision and image processing, the existence of error bounds is a unique feature for compressive sensing problems. What is more, the proof of these stunning results is actually not super-hard, and any (motivated) graduate or undergraduate student with a basic knowledge of properties of vectors, and (more than) a little bit of patience, can easily follow the proofs. The purpose of this exercise is to take you through the proof of the following theorem proved by Emmanuel Candes: \textsf{[12 points]}
\\
\textbf{Theorem:} Let $\mathbf{y} = \mathbf{Ax}+\mathbf{\eta}$ be a vector of noisy compressed measurements where the matrix $\mathbf{A} \in \mathbb{R}^{m \times n},  m \ll n$ has restricted isometry constant $\delta_{2s} < \sqrt{2}-1$. Let the noise magnitude be upper bounded as $\|\mathbf{\eta}\|_2 \leq \epsilon$. Let $\mathbf{x}^{\star}$ be the solution to the problem $\textrm{min}_{\mathbf{x}} \|\mathbf{x}\|_1$ such that $\|\mathbf{y} - \mathbf{Af}\|_2 \leq \epsilon$. Then $\mathbf{x^{\star}}$ obeys 
$\|\mathbf{x^{\star} - x}\|_2 \leq C_0 s^{-1/2}\|\mathbf{x - x_s}\|_1  + C_1 \epsilon$ where $C_0$ and $C_1$ are small-valued constants and $\mathbf{x_s}$ is a vector obtained by nullifying all entries of $\mathbf{x}$ except the $s$ entries with the largest absolute value. 
\\
\\
The proof is given below. Your job is to trace through it, verifying every step, and then answering the questions presented in blue colored font. \emph{Ideally, edit the latex file itself and write your answer in blue colored font}.
You will need to use the triangle inequality ($\|\mathbf{x}\|_2 + \|\mathbf{y}\|_2 \geq \|\mathbf{x}+\mathbf{y}\|_2$), the reverse triangle inequality ($\|\mathbf{x}-\mathbf{y}\|_2 \geq \|\mathbf{x}\|_2 - \|\mathbf{y}\|_2$), the Cauchy-Schwartz inequality (the dot product $|\langle \mathbf{x}, \mathbf{y} \rangle| \leq \|\mathbf{x}\|_2 \|\mathbf{y}\|_2$) for vectors $\mathbf{x}$ and $\mathbf{y}$, and the restricted isometry property for $\mathbf{A}$. Also remember that $\|\mathbf{x}\|_1 = \sum_i |x_i|$, $\|\mathbf{x}\|_2 = \sqrt{\sum_i x^2_i}$ and $\|\mathbf{x}\|_{\infty} = \textrm{max}_i |x_i|$.
\\
\\
\textbf{Proof:}
\\
\begin{enumerate}
\item We can write the following result: $\|\mathbf{A(x-x^{\star})}\|_2 \leq 2\epsilon$. \textcolor{blue}{(How is this derived?)}
\begin{eqnarray}\label{part1}
\|\mathbf{A(x-x^{\star})}\|_2 = \|\mathbf{Ax-Ax^{\star}}\|_2 = \|\mathbf{y - \eta Ax^{\star}}\|_2 =  \|\mathbf{y - \eta -Ax^{\star}}\|_2 \leq \|\mathbf{y -Ax^{\star}}\|_2 + \|\eta\|_2 = 2\epsilon
\end{eqnarray}
\item Let us define vector $\mathbf{h} = \mathbf{x^{\star}-x}$. Let us denote vector $\mathbf{h_T}$ as the vector equal to $\mathbf{h}$ only on an index set $T$ and zero at all other indices. Let $T_0$ the set of indices containing the $s$ largest entries of $\mathbf{x}$ (in terms of absolute value), $T_1$ be the set of indices of the $s$ largest entries of \cancel{$\mathbf{x}$} \textcolor{red}{$\mathbf{h}_{T^c_0}$}, $T_2$ be the set of indices of the next $s$ largest entries of \cancel{$\mathbf{x}$}\textcolor{red}{$\mathbf{h}_{T^c_0}$} after $T_1$, and so on. We will now decompose $\mathbf{h}$ as the sum of $\mathbf{h_{T_0}}, \mathbf{h_{T_1}}, \mathbf{h_{T_2}}, ...$. We will denote the complement of an index set $T$ as $T^c$. Our aim will be to prove that both $\|\mathbf{h_{T_0 \cup T_1}}\|_2$ and $\|\mathbf{h_{(T_0 \cup T_1)^c}}\|_2$ are upper bounded by sensible and intuitive quantities. 
\item We will first prove the bound on $\|\mathbf{h}_{(T_0 \cup T_1)^c}\|_2$, in the following way, using simple vector properties and the fact that $\mathbf{x+h}$ is the minimum of the objective function subject to the given constraint. 
\begin{enumerate}
\item We have $\|\mathbf{h}_{T_j}\|_2 \leq s^{1/2} \|\mathbf{h}_{T_j}\|_\infty \leq s^{-1/2} \|\mathbf{h}_{T_{j-1}}\|_1$. \textcolor{blue}{(Prove both these inequalities. Note that any element of $T_{j-1}$ is greater than or equal to any element of $T_j$ for any $j \geq 1$)}. \\
Consider,
\begin{eqnarray} \label{part3a}
\sum\limits_{i\in T_{j}} h_{i}^{2} \leq  s\max_{i \in T_{j}} h_{i}^2
\implies \|\mathbf{h}_{T_j}\|_2 \leq s^{1/2} \|\mathbf{h}_{T_j}\|_\infty\\
\text{For every term in $h_{T_{j}}$ there exists a corresponding term in $h_{T_{j-1}}$ such that}
|\mathbf{h}_{T_j,i}| \leq |\mathbf{h}_{T_{j-1,i}}|\\
\forall i \in T_{j-1} \max_{k \in T_{j}} h_{T_{j},k} \leq h_{T_{j-1},i} \implies s \|\mathbf{h}_{T_j}\|_\infty \leq \|\mathbf{h}_{T_{j-1}}\|_1 \implies s^{1/2} \|\mathbf{h}_{T_j}\|_\infty \leq s^{-1/2}\|\mathbf{h}_{T_{j-1}}\|_1
\end{eqnarray}
\item Therefore $\sum_{j \geq 2}\|\mathbf{h}_{T_j}\|_2 \leq s^{-1/2} \|\mathbf{h}_{(T_0)^c}\|_1$. \textcolor{blue}{(Prove this inequality)}. 
\begin{eqnarray} \label{part3b}
\|\mathbf{h}_{(T_0)^c}\|_1 = \sum\limits_{j \geq 1} \|\mathbf{h}_{(T_{j}}\|_1
\implies \text{ using ~\ref{part3a}} \quad \sum_{j \geq 2}\|\mathbf{h}_{T_j}\|_2 \leq s^{-1/2} \|\mathbf{h}_{(T_0)^c}\|_1
\end{eqnarray}
\item Hence we have $\|\mathbf{h}_{(T_0 \cup T_1)^c}\|_2 = \|\sum_{j \geq 2} \mathbf{h}_{T_j}\|_2 \leq \sum_{j \geq 2} \|\mathbf{h}_{T_j}\|_2 \leq s^{-1/2} \|\mathbf{h}_{(T_0)^c}\|_1$ (\textcolor{blue}{Prove both inequalities}).
\begin{eqnarray}\label{part3c}
\|\sum_{j \geq 2} \mathbf{h}_{T_j}\|_2 \leq \sum_{j \geq 2} \|\mathbf{h}_{T_j}\|_2 \quad \text{Cauchy-Schwarz Inequality}\\
\sum_{j \geq 2} \|\mathbf{h}_{T_j}\|_2 \leq s^{-1/2} \|\mathbf{h}_{(T_0)^c}\|_1 \quad \text {using ~\ref{part3b}}
\end{eqnarray}
\item Now it turns out that $\|\mathbf{h}_{(T_0)^c}\|_1$ is not very large in value. Why so? As $\mathbf{x}^{\star}$ is the minimum, we have $\|\mathbf{x}\|_1 \geq \|\mathbf{x}+\mathbf{h}\|_1 = \sum_{i \in T_0} |x_i + h_i| + \sum_{i \in {(T_0)}^c} |x_i + h_i| \geq \|\mathbf{x}_{T_0}\|_1 - \|\mathbf{h}_{T_0}\|_1 + \|\mathbf{h}_{{(T_0)}^c}\|_1 - \|\mathbf{x}_{{(T_0)^c}}\|_1$ \textcolor{blue}{Prove the last inequality}. 
\begin{eqnarray}
\sum_{i \in T_0} |x_i + h_i| \geq  \|\mathbf{x}_{T_0}\|_1 - \|\mathbf{h}_{T_0}\|_1 \quad \text {using reverse triangle inequality on each component of x and h}\\
\sum_{i \in {(T_0)}^c} |x_i + h_i| \geq \|\mathbf{h}_{{(T_0)}^c}\|_1 - \|\mathbf{x}_{{(T_0)^c}}\|_1 \quad \text{using reverse triangle inequality on each component of x and h}\\
\therefore \|\mathbf{x}\|_1 \geq \|\mathbf{x}+\mathbf{h}\|_1 = \sum_{i \in T_0} |x_i + h_i| + \sum_{i \in {(T_0)}^c} |x_i + h_i| \geq \|\mathbf{x}_{T_0}\|_1 - \|\mathbf{h}_{T_0}\|_1 + \|\mathbf{h}_{{(T_0)}^c}\|_1 - \|\mathbf{x}_{{(T_0)^c}}\|_1
\end{eqnarray}
\item Rearranging the terms now gives us $\|\mathbf{h}_{{(T_0)}^c}\|_1 \leq \|\mathbf{h}_{{(T_0)}}\|_1  + 2\|\mathbf{x}_{{(T_0)^c}}\|_1 = \|\mathbf{h}_{{(T_0)}}\|_1  + 2\|\mathbf{x}-\mathbf{x_s}\|_1$. 
\begin{eqnarray}\label{part3e}
\|\mathbf{h}_{{(T_0)}^c}\|_1 \leq \|\mathbf{h}_{{(T_0)}}\|_1  + 2\|\mathbf{x}_{{(T_0)^c}}\|_1 = \|\mathbf{h}_{{(T_0)}}\|_1  + 2\|\mathbf{x}-\mathbf{x_s}\|_1
\end{eqnarray}
\item Combining everything, we now have $\|\mathbf{h}_{(T_0 \cup T_1)^c}\|_2 \leq s^{-1/2}(\|\mathbf{h}_{{(T_0)}}\|_1  + 2\|\mathbf{x}-\mathbf{x_s}\|_1) \leq \|\mathbf{h}_{{(T_0)}}\|_2 + \textcolor{red}{2}s^{-1/2} \|\mathbf{x}-\mathbf{x_s}\|_1$. \textcolor{blue}{(Prove the last inequality).}
\begin{eqnarray}\label{part3f}
\|\mathbf{h}_{(T_0 \cup T_1)^c}\|_2 \leq s^{-1/2}(\|\mathbf{h}_{{(T_0)}}\|_1  + 2\|\mathbf{x}-\mathbf{x_s}\|_1) \text{using ~\ref{part3c} and ~\ref{part3e}}\\
\dfrac{\|\mathbf{h}_{{(T_0)}}\|_1}{s} \leq s^{-1/2}\|\mathbf{h}_{{(T_0)}}\|_2 \implies s^{-1/2}\|\mathbf{h}_{{(T_0)}}\|_1 \leq \|\mathbf{h}_{{(T_0)}}\|_2\\
\therefore s^{-1/2}(\|\mathbf{h}_{{(T_0)}}\|_1  + 2\|\mathbf{x}-\mathbf{x_s}\|_1) \leq \|\mathbf{h}_{{(T_0)}}\|_2
\end{eqnarray}
\end{enumerate}
\item We will now prove the bound on $\|\mathbf{h}_{(T_0 \cup T_1)}\|_2$, in the following way. 
\begin{enumerate}
\item We observe that $\mathbf{Ah_{T_0 \cup T_1}} = \mathbf{Ah} - \sum_{j \geq 2} \mathbf{Ah_{T_j}}$. \\
Hence $\|\mathbf{Ah_{T_0 \cup T_1}}\|^2_2 = \langle \mathbf{Ah_{T_0 \cup T_1}} , \mathbf{Ah}\rangle - \langle \mathbf{Ah_{T_0 \cup T_1}} , \sum_{j \geq 2} \mathbf{Ah_{T_j}}\rangle$.
\begin{eqnarray}\label{part4a}
\|\mathbf{Ah_{T_0 \cup T_1}}\|^2_2 = \langle \mathbf{Ah_{T_0 \cup T_1}} , \mathbf{Ah}\rangle - \langle \mathbf{Ah_{T_0 \cup T_1}} , \sum_{j \geq 2} \mathbf{Ah_{T_j}}\rangle
\end{eqnarray}
\item Now, we have $|\langle \mathbf{Ah}_{T_0 \cup T_1} , \mathbf{Ah}\rangle| \leq \|\mathbf{Ah}_{T_0 \cup T_1}\|_2 \|\mathbf{Ah}\|_2 \leq 2 \epsilon \sqrt{1 + \textcolor{red}{\cancel{2}}\delta_{2s}} \|\mathbf{h}_{T_0 \cup T_1}\|_2$ \textcolor{blue}{(Prove both these inequalities, one of which uses the restricted isometry property of $\mathbf{A}$)}.
\begin{eqnarray}\label{part4b}
|\langle \mathbf{Ah}_{T_0 \cup T_1} , \mathbf{Ah}\rangle| \leq \|\mathbf{Ah}_{T_0 \cup T_1}\|_2 \|\mathbf{Ah}\|_2 \quad \text{using Cauchy-Schwarz Inequality}\\
\|\mathbf{Ah}_{T_0 \cup T_1}\|_2 \|\mathbf{Ah}\|_2 \leq 2 \epsilon \sqrt{1 + \delta_{2s}} \|\mathbf{h}_{T_0 \cup T_1}\|_2 \quad \text{using Restricted Isometery Property $h = x-x\star$ and ~\ref{part1}}
\end{eqnarray}
\item We also have $|\langle \mathbf{Ah}_{T_0}, \mathbf{Ah}_{T_j}\rangle| \leq \delta_{2s} \|\mathbf{h}_{T_0}\|_2 \|\mathbf{h}_{T_j}\|_2$. To prove this, observe that $\mathbf{h_{T_0}}$ and $\mathbf{h_{T_j}}$ are vectors with disjoint support. $|\langle \mathbf{Ah}_{T_0}, \mathbf{Ah}_{T_j}\rangle| = \|\mathbf{h}_{T_0}\|_2 \|\mathbf{h}_{T_j}\|_2|\langle \mathbf{A\hat{h}}_{T_0}, \mathbf{A\hat{h}}_{T_j}\rangle|$  where $\mathbf{\hat{h}_{T_0}}$ and $\mathbf{\hat{h}_{T_j}}$ are unit-normalized vectors. This further yields $|\langle \mathbf{Ah}_{T_0}, \mathbf{Ah}_{T_j}\rangle|=\|\mathbf{h}_{T_0}\|_2 \|\mathbf{h}_{T_j}\|_2 \frac{\|\mathbf{A}(\mathbf{\hat{h}_{T_0}}+\mathbf{\hat{h}_{T_j}})\|^2 - \|\mathbf{A}(\mathbf{\hat{h}_{T_0}}-\mathbf{\hat{h}_{T_j}})\|^2}{4} 
\\ \leq \|\mathbf{h}_{T_0}\|_2 \|\mathbf{h}_{T_j}\|_2 \frac{(1+\delta_{2s})(\|\mathbf{h_{T_0}}\|^2+\|\mathbf{h_{T_j}}\|^2)-(1-\delta_{2s})(\|\mathbf{h_{T_0}}\|^2+\|\mathbf{h_{T_j}}\|^2)}{4} \\
= \delta_{2s} \|\mathbf{h}_{T_0}\|_2 \|\mathbf{h}_{T_j}\|_2$. Analogously, $|\langle \mathbf{Ah}_{T_1}, \mathbf{Ah}_{T_j}\rangle| \leq \delta_{2s} \|\mathbf{h}_{T_1}\|_2 \|\mathbf{h}_{T_j}\|_2$. 
\begin{equation}\label{part4c}
|\langle \mathbf{Ah}_{T_0}, \mathbf{Ah}_{T_j}\rangle| \leq \delta_{2s} \|\mathbf{h}_{T_0}\|_2 \|\mathbf{h}_{T_j}\|_2
\end{equation}
\item Now, we have $(1-\delta_{2s})\|\mathbf{h}_{T_0 \cup T_1}\|^2_2 
\leq \|\mathbf{Ah}_{T_0 \cup T_1}\|^2_2 \leq 2 \epsilon \sqrt{1+\delta_{2s}}\|\mathbf{h}_{T_0 \cup T_1}\|_2 + \delta_{2s}(\|\mathbf{h}_{T_0}\|_2 + \|\mathbf{h}_{T_1}\|_2) \sum_{j \geq 2} \|\mathbf{h}_{T_j}\|_2$. \textcolor{blue}{(Prove this).} Furthermore, we can write $(1-\delta_{2s})\|\mathbf{h}_{T_0 \cup T_1}\|^2_2  \leq \|\mathbf{h}_{T_0 \cup T_1}\|_2(2\epsilon\sqrt{1+2\delta_{2s}}) + \sqrt{2}\delta_{2s} \sum_{j \geq 2} \|\mathbf{h}_{T_j}\|_2)$ because $\mathbf{h}_{T_0}$ and $\mathbf{h}_{T_1}$ are vectors with disjoint sets of non-zero indices and hence $\|\mathbf{h}_{T_0}\|_2 + \|\mathbf{h}_{T_1}\|_2 \leq \sqrt{2} \|\mathbf{h}_{T_0 \cup T_1}\|_2$.
\begin{equation}\label{part4d}
\begin{aligned}
(1-\delta_{2s})\|\mathbf{h}_{T_0 \cup T_1}\|^2_2 
 & \leq  \|\mathbf{Ah}_{T_0 \cup T_1}\|^2_2  \quad \text{using the RIP property}\\
\|\mathbf{Ah}_{T_0 \cup T_1}\|^2_2 &= |\langle \mathbf{Ah}_{T_0 \cup T_1} , \mathbf{Ah} - \mathbf{Ah}_{(T_{0} \cup T_{1})^c} \rangle|\\ &=  |\langle \mathbf{Ah}_{T_0 \cup T_1} , \mathbf{Ah}\rangle| - |\langle \mathbf{Ah}_{T_0 \cup T_1} , \mathbf{Ah}_{(T_{0} \cup T_{1})^c} \rangle|\\
&\leq |\langle \mathbf{Ah}_{T_0 \cup T_1} , \mathbf{Ah}\rangle| + | |\langle \mathbf{Ah}_{T_0 \cup T_1} , \mathbf{Ah}_{(T_{0} \cup T_{1})^c} \rangle| |\\ &= 2 \epsilon \sqrt{1+\delta_{2s}}\|\mathbf{h}_{T_0 \cup T_1}\|_2  + \delta_{2s} \sum\limits_{j\geq 2} \|\mathbf{h}_{T_j}\|_2 \|\mathbf{h}_{T_0}\|_2 + \delta_{2s} \sum\limits_{j\geq 2} \|\mathbf{h}_{T_j}\|_2 \|\mathbf{h}_{T_1}\|_2\\
&= 2 \epsilon \sqrt{1+\delta_{2s}}\|\mathbf{h}_{T_0 \cup T_1}\|_2 + \delta_{2s}(\|\mathbf{h}_{T_0}\|_2 + \|\mathbf{h}_{T_1}\|_2) \sum_{j \geq 2} \|\mathbf{h}_{T_j}\|_2\\
& \text{using ~\ref{part4c} and ~\ref{part4b}}
\end{aligned}
\end{equation}
\item Rearranging the above equation, and using one of the previous results (\textcolor{blue}{which one? = ~\ref{part3f}}), 
we get $\|\mathbf{h}_{T_0 \cup T_1}\|_2 \leq \epsilon \dfrac{2\sqrt{1+\delta_{2s}}}{1-\delta_{2s}} + \dfrac{\sqrt{2}\sqrt{\delta_{2s}}}{1-\delta_{2s}} s^{-1/2} \|\mathbf{h}_{(T_0)^c}\|_1 
\\ \leq \epsilon \dfrac{2\sqrt{1+\delta_{2s}}}{1-\delta_{2s}} + \dfrac{\sqrt{2}\textcolor{red}{\cancel{\sqrt{\delta_{2s}}}} \textcolor{red}{\delta_{2s}}}{1-\delta_{2s}} \|\mathbf{h}_{T_0 \cup T_1}\|_2 + \dfrac{\sqrt{2}\delta_{2s}\textcolor{red}{\cancel{\sqrt{\delta_{2s}}}}}{1-\delta_{2s}} s^{-1/2} \|\mathbf{x}-\mathbf{x_s}\|_1$. Further rearranging gives us
$\|\mathbf{h}_{T_0 \cup T_1}\|_2  \leq C' \epsilon + C'' s^{-1/2} \|\mathbf{x}-\mathbf{x_s}\|_1$ where $C'$ and $C''$ are constants that depend only on $\delta_{2s}$ (note in particular that they do not depend on $n$).
\begin{equation}\label{part4e}
\begin{aligned}
\|\mathbf{h}_{T_0 \cup T_1}\|_2 & \leq \epsilon \dfrac{2\sqrt{1+\delta_{2s}}}{1-\delta_{2s}} + \dfrac{\sqrt{2}\sqrt{\delta_{2s}}}{1-\delta_{2s}} s^{-1/2} \|\mathbf{h}_{(T_0)^c}\|_1 
\\ & \leq \epsilon \dfrac{2\sqrt{1+\delta_{2s}}}{1-\delta_{2s}} + \dfrac{\sqrt{2}\delta_{2s}}{1-\delta_{2s}} \|\mathbf{h}_{T_0 \cup T_1}\|_2 + \dfrac{\sqrt{2}\delta_{2s}}{1-\delta_{2s}} s^{-1/2} \|\mathbf{x}-\mathbf{x_s}\|_1\\ &\leq \epsilon \dfrac{2\sqrt{1+\delta_{2s}}}{1-\delta_{2s}} \dfrac{1-\delta_{2s}}{1-\delta_{2s} -\sqrt{2}\delta_{2s}} +  \dfrac{\sqrt{2}\delta_{2s}}{1-\delta_{2s}} s^{-1/2}\dfrac{1-\delta_{2s}}{1-\delta_{2s} -\sqrt{2}\delta_{2s}} \|\mathbf{x}-\mathbf{x_s}\|_1\\
&=\epsilon \dfrac{2\sqrt{1+\delta_{2s}}}{1-\delta_{2s} -\sqrt{2}\delta_{2s}} + s^{-1/2} \dfrac{\sqrt{2}\delta_{2s}}{1-\delta_{2s} -\sqrt{2}\delta_{2s}} \|\mathbf{x}-\mathbf{x_s}\|_1
\end{aligned}
\end{equation}
\end{enumerate}
\item Now, we combine the upper bounds on $\|\mathbf{h}_{(T_0 \cup T_1)}\|_2$ and $\|\mathbf{h}_{(T_0 \cup T_1)^c}\|_2$ to yield the final result as follows: \\
$\|\mathbf{h}\|_2 \leq \|\mathbf{h}_{T_0 \cup T_1}\|_2 + \|\mathbf{h}_{(T_0 \cup T_1)^c}\|_2 \leq \|\mathbf{h}_{T_0 \cup T_1}\|_2 + \|h_{T_0}\|_2 + 2s^{-1/2}\|\mathbf{x}-\mathbf{x_s}\|_1 \\
\leq 2 \|\mathbf{h}_{T_0 \cup T_1}\|_2 + 2\textcolor{red}{s^{-1/2}}\|\mathbf{x}-\mathbf{x_s}\|_1 \leq C_0 s^{-1/2} \|\mathbf{x}-\mathbf{x_s}\|_1 + C_1 \epsilon$ \textcolor{blue}{(Justify all these inequalities. Write the final expression for $C_0$ and $C_1$ and explain where the sufficient condition $\delta_{2s} \leq \sqrt{2}-1$ arises)}.
\begin{eqnarray}
\|\mathbf{h}\|_2 \leq \|\mathbf{h}_{T_0 \cup T_1}\|_2 + \|\mathbf{h}_{(T_0 \cup T_1)^c}\|_2 \text{cauchy  schwarz}\\
\|\mathbf{h}_{T_0 \cup T_1}\|_2 + \|\mathbf{h}_{(T_0 \cup T_1)^c}\|_2 \leq \|\mathbf{h}_{T_0 \cup T_1}\|_2 + \|h_{T_0}\|_2 + 2s^{-1/2}\|\mathbf{x}-\mathbf{x_s}\|_1 \text {using ~\ref{part3f}}\\
\|\mathbf{h}_{T_0 \cup T_1}\|_2 + \|h_{T_0}\|_2 + 2s^{-1/2}\|\mathbf{x}-\mathbf{x_s}\|_1 \\
\leq  \|\mathbf{h}_{T_0 \cup T_1}\|_2+ \|\mathbf{h}_{T_0 \cup T_1}\|_2 + 2\textcolor{red}{s^{-1/2}}\|\mathbf{x}-\mathbf{x_s}\|_1\\ 
2 \|\mathbf{h}_{T_0 \cup T_1}\|_2 + 2\textcolor{red}{s^{-1/2}}\|\mathbf{x}-\mathbf{x_s}\|_1 \leq C_0 s^{-1/2} \|\mathbf{x}-\mathbf{x_s}\|_1 + C_1 \epsilon \quad \text{using ~\ref{part4e}}\\
C_{0} = \dfrac{2\sqrt{2}\delta_{2s}}{1-\delta_{2s} -\sqrt{2}\delta_{2s}} + 2\\
C_{1} = \dfrac{2*2\sqrt{1+\delta_{2s}}}{1-\delta_{2s} -\sqrt{2}\delta_{2s}}
\end{eqnarray}
\end{enumerate}

\end{document}