\title{Assignment 1: Question 2}
\author{}
\date{}

\documentclass[11pt]{article}

\usepackage{amsmath}
\usepackage{amssymb}
\usepackage{hyperref}
\usepackage{ulem}
\usepackage[margin=0.5in]{geometry}
\setlength\parindent{0pt}

\begin{document}
\maketitle

\textbf{Question:} In this exercise, we will prove the orthocenter theorem pertaining to the vanishing points $Q,R,S$ of three mutually perpendicular directions $OQ, OR, OS$, where $O$ is the pinhole (origin of camera coordinate system). Let the image plane be $Z = f$. Recall that two directions $v_1$ and $v_2$ are orthogonal if $v^T_1 v_2 = 0$. One can conclude that $OS$ is orthogonal to $OR-OQ$ (why?). Also the optical axis $Oo$ (where $o$ is the optical center) is orthogonal to $OR-OQ$ (why?). Hence the plane formed by triangle $OSo$ is orthogonal to $OR-OQ$ and hence line $oS$ is perpendicular to $OR-OQ = QR$ (why?). Likewise $oR$ and $oQ$ are perpendicular to $QS$ and $RS$. Hence we have proved that the altitudes of the triangle $QRS$ are concurrent at the point $o$. QED. Now, in this proof, I considered the three perpendicular lines to be passing through $O$. How will you modify the proof if the three lines did not pass through $O$? \textsf{[4 points]} \\

\textbf{Answer:} \\
Since $OR$ and $OQ$ are orthogonal to $OS$,
\begin{eqnarray*}
OS^TOR &=& 0
\end{eqnarray*}
and
\begin{eqnarray*}
OS^TOQ &=& 0
\end{eqnarray*}
Consider,
\begin{eqnarray*}
OS^T(OR - OQ) &=& OS^TOR - OS^TOQ\\
&=& 0
\end{eqnarray*}
Therefore, $OS$ is orthogonal to $OR-OQ$.\\

$Q$ and $R$ lie on the image plane and the optical axis $Oo$ is perpendicular to the image plane.\\
Therefore, $Oo$ is perpendicular to $OR-OQ = QR$.\\

This implies that the plane defined by $Oo$ and $OS$ is orthogonal to $OR - OQ$. This is the plane formed by the triangle $OSo$.\\
Therefore, $oS$ is perpendicular to $OR - OQ = QR$.\\

Similarly, $oR$ and $oQ$ are perpendicular to $QS$ and $RS$. \\
Hence altitudes of the triangle $QRS$ are concurrent at the point $o$.\\

Let $\ell_1, \ell_2$ and $\ell_3$ be three perpendicular lines not passing through the pinhole $O$. \\
We can translate such that they pass through the origin, this will not change their vanishing points. ($\because$ parallel lines have same vanishing point). \\
We can now apply the above procedure to prove the orthocenter property. \\

\end{document}