\title{Assignment 3: Question 2}
\author{}

\documentclass[11pt]{article}

\usepackage{amsmath}
\usepackage{amssymb}
\usepackage{hyperref}
\usepackage{ulem}
\usepackage[margin=0.5in]{geometry}
\begin{document}
\maketitle

A Lambertian object illuminated by a point source has a reflectance map of the form given by 
\begin{equation}
R(p,q) = \frac{1+pp_s + qq_s}{\sqrt{1+p^2_s+q^2_s}\sqrt{1+p^2+q^2}}
\end{equation}
where the surface normal is $(-p,-q,1)$ and the light source direction is $(-p_s,-q_s,1)$. What value(s) of $(p,q)$ will maximize $R(p,q)$? For what values, will you get $R(p,q) = 0$? \textsf{[2 points]}\\

Answer: \\
\begin{equation}
R(p,q) = \frac{1+pp_s + qq_s}{\sqrt{1+p^2_s+q^2_s}\sqrt{1+p^2+q^2}}
\end{equation}
where the surface normal is $(-p,-q,1)$ and the light source direction is $(-p_s,-q_s,1)$. \\

Then, \\
$ R(p,q) = cos (\theta) $ where $\theta$ is the angle between surface normal and the light source direction. \\
Maximum value of $ R(p,q)$ occurs when $ cos (\theta) = 1$, that is, $(-p,-q,1)$ and $(-p_s,-q_s,1)$ are parallel. \\
Therefore, $ p = p_{s} $ and $ q = q_{s}$.

$ R(p,q) = 0 $ implies $\theta = 90^{\circ}$. \\
$ 1 + pp_s + qq_s = 0$ \\
$ p = - (1 + qq_s)/p_s$ \\
Therefore, $ R(p,q) = 0 $ for any pair ($- (1 + qq_s)/p_s, q$) for any $q$. \\

\end{document}